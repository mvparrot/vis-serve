\documentclass[article]{jss}
\usepackage[utf8]{inputenc}

\providecommand{\tightlist}{%
  \setlength{\itemsep}{0pt}\setlength{\parskip}{0pt}}

\author{
Alwin Wang\\Monash University
}
\title{Visualising Serve Trajectories in High-Performance Tennis with R}
\Keywords{keywords, not capitalized, \proglang{Java}}

\Abstract{
This paper investigates methods to effectively visualise key
characteristics of serves using trajectory of elite tennis athletes
provided by Tennis Australia. The key characteristics identified were
the position, velocity and spin of a ball as well as the location of
single and multiple serve clusters. For the visuals presented in the
paper, a sample of 2000 serves from the 2016 Australian Open from thirty
three servers was used.
}

\Plainauthor{Alwin Wang}
\Plaintitle{Visualising Serve Trajectories in High-Performance Tennis with R}
\Shorttitle{Visualising Serves}
\Plainkeywords{keywords, not capitalized, Java}

%% publication information
%% \Volume{50}
%% \Issue{9}
%% \Month{June}
%% \Year{2012}
\Submitdate{}
%% \Acceptdate{2012-06-04}

\Address{
    Alwin Wang\\
  Monash University\\
  First line Second line\\
  E-mail: \href{mailto:awan39@student.monash.edu}{\nolinkurl{awan39@student.monash.edu}}\\
  URL: \url{http://rstudio.com}\\~\\
  }

\usepackage{amsmath}

\begin{document}

\section{Introduction}\label{introduction}

This template demonstrates some of the basic latex you'll need to know
to create a JSS article.

\subsection{Code formatting}\label{code-formatting}

Don't use markdown, instead use the more precise latex commands:

\begin{itemize}
\tightlist
\item
  \proglang{Java}
\item
  \pkg{plyr}
\item
  \code{print("abc")}
\end{itemize}

\section{R code}\label{r-code}

Can be inserted in regular R markdown blocks.

\begin{CodeChunk}
\begin{CodeInput}
R> x <- 1:10
R> x
\end{CodeInput}
\begin{CodeOutput}
 [1]  1  2  3  4  5  6  7  8  9 10
\end{CodeOutput}
\end{CodeChunk}



\end{document}

